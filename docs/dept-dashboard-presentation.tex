% Dept-Dashboard — Beamer presentation
% Build (from repo root):
%   pdflatex -interaction=nonstopmode -halt-on-error -output-directory docs docs/dept-dashboard-presentation.tex
%   pdflatex -interaction=nonstopmode -halt-on-error -output-directory docs docs/dept-dashboard-presentation.tex
% Build (from docs/):
%   cd docs && pdflatex -interaction=nonstopmode -halt-on-error dept-dashboard-presentation.tex
\documentclass[aspectratio=169]{beamer}

\usetheme{Madrid}
\setbeamertemplate{navigation symbols}{}
\setbeamertemplate{caption}[numbered]

\usepackage[T1]{fontenc}
\usepackage[utf8]{inputenc}
\usepackage[french]{babel}
\usepackage{lmodern}
\usepackage{microtype}
\usepackage{graphicx}
\usepackage{booktabs}
\usepackage{tabularx}

% Make screenshots work whether compiled from repo root or from docs/
\graphicspath{{../screenshots/}{screenshots/}}

\definecolor{DDPrimary}{HTML}{0B5394}
\definecolor{DDAccent}{HTML}{009688}
\definecolor{DDLight}{HTML}{F2F5F9}

\setbeamercolor{structure}{fg=DDPrimary}
\setbeamercolor{title}{fg=DDPrimary}
\setbeamercolor{subtitle}{fg=DDAccent}
\setbeamercolor{frametitle}{bg=DDPrimary, fg=white}
\setbeamercolor{block title}{bg=DDLight, fg=DDPrimary}
\setbeamercolor{block body}{bg=white, fg=black}
\setbeamercolor{author in head/foot}{bg=DDLight, fg=DDPrimary}
\setbeamercolor{date in head/foot}{bg=DDLight, fg=DDPrimary}

\setbeamerfont{title}{size=\LARGE, series=\bfseries}
\setbeamerfont{frametitle}{size=\large, series=\bfseries}
\setbeamerfont{block title}{series=\bfseries}

\setbeamersize{text margin left=8mm, text margin right=8mm}
\setbeamertemplate{blocks}[rounded][shadow=false]
\setbeamertemplate{headline}{}
\setbeamertemplate{itemize item}{\color{DDAccent}$\blacktriangleright$}
\setbeamertemplate{itemize subitem}{\color{DDAccent}\scriptsize$\blacktriangleright$}

\setbeamertemplate{footline}{%
  \leavevmode%
  \hbox{%
    \begin{beamercolorbox}[wd=.78\paperwidth,ht=2.7ex,dp=1.2ex,left]{author in head/foot}%
      \hspace{0.8em}\usebeamerfont{author in head/foot}\insertshorttitle%
      \hspace{1.0em}{\color{DDAccent}\textbar}\hspace{1.0em}\insertsectionhead%
    \end{beamercolorbox}%
    \begin{beamercolorbox}[wd=.22\paperwidth,ht=2.7ex,dp=1.2ex,right]{date in head/foot}%
      \usebeamerfont{date in head/foot}\insertframenumber/\inserttotalframenumber\hspace{0.8em}%
    \end{beamercolorbox}%
  }%
  \vskip0pt%
}

\setbeamertemplate{section page}{%
  \vspace{2.0cm}
  \begin{center}
    {\usebeamerfont{title}\color{DDPrimary}\insertsectionhead\par}
    \vspace{0.6em}
    {\color{DDAccent}\rule{0.62\linewidth}{0.7pt}}
  \end{center}
}

\AtBeginSection{
  \begin{frame}[plain]
    \sectionpage
  \end{frame}
}

\title[Dept-Dashboard]{Dept-Dashboard}
\subtitle{Tableau de bord modulaire pour départements d'enseignement (IUT)}
\author{}
\institute{}
\date{\today}

\begin{document}

\begin{frame}[plain]
  \begin{columns}[T]
    \column{0.55\textwidth}
      \vspace{0.6cm}
      {\usebeamerfont{title}\color{DDPrimary}\inserttitle\par}
      \vspace{0.3em}
      {\usebeamerfont{subtitle}\color{DDAccent}\insertsubtitle\par}
      \vspace{1.0em}
      \small
      \begin{itemize}
        \item KPIs scolarité, recrutement, budget, EDT.
        \item Alertes réussite \& fiche étudiant.
        \item Multi-départements, permissions, cache Redis.
      \end{itemize}
      \vfill
      \footnotesize \insertdate
    \column{0.45\textwidth}
      \vspace{0.2cm}
      \includegraphics[width=\textwidth]{main.png}
  \end{columns}
\end{frame}

\begin{frame}{Plan}
  \tableofcontents[hideallsubsections]
\end{frame}

\section{Vision \& périmètre}

\begin{frame}{Contexte \& objectifs}
  \begin{itemize}
    \item Centraliser plusieurs sources (ScoDoc, Parcoursup, fichiers Excel/CSV).
    \item Visualiser des indicateurs clés (KPIs) : scolarité, recrutement, budget, EDT/maquettes.
    \item Sécuriser l'accès (CAS + JWT) et gérer des droits par département.
    \item Accélérer l'accès à l'information (cache Redis + rafraîchissement contrôlé).
  \end{itemize}
\end{frame}

\begin{frame}{Fonctionnalités principales}
  \small
  \begin{tabularx}{\textwidth}{@{}lX@{}}
    \toprule
    \textbf{Module} & \textbf{Ce que l'on suit} \\
    \midrule
    Scolarité & Effectifs, résultats, progression, assiduité (si disponible) \\
    Réussite / Alertes & Détection d'étudiants en difficulté, absentéisme, risque de décrochage \\
    Recrutement & Statistiques Parcoursup, profils candidats, suivi de campagne \\
    Budget & Suivi des dépenses, répartition par catégories, alertes de seuil \\
    EDT \& Maquettes & Charges enseignantes, occupation salles, répartition CM/TD/TP \\
    Administration & Utilisateurs, permissions granulaires multi-départements \\
    \bottomrule
  \end{tabularx}
\end{frame}

\section{Parcours utilisateur}

\begin{frame}{Connexion (CAS / mode dev)}
  \begin{columns}[T]
    \column{0.52\textwidth}
      \begin{block}{Points clés}
        \begin{itemize}
          \item En production : authentification via CAS.
          \item En développement : mode mock (connexion rapide).
          \item Après connexion : accès filtré selon permissions.
        \end{itemize}
      \end{block}
    \column{0.48\textwidth}
      \includegraphics[width=\textwidth]{login.png}
  \end{columns}
\end{frame}

\begin{frame}{Tableau de bord (vue d'ensemble)}
  \begin{columns}[T]
    \column{0.62\textwidth}
      \includegraphics[width=\textwidth]{home.png}
    \column{0.38\textwidth}
      \begin{block}{À retenir}
        \small
        \begin{itemize}
          \item Synthèse par domaine.
          \item Filtres (département, période/année selon module).
          \item Exports pour le pilotage.
        \end{itemize}
      \end{block}
  \end{columns}
\end{frame}

\section{Modules métier}

\subsection{Scolarité}
\begin{frame}{Module Scolarité}
  \begin{columns}[T]
    \column{0.54\textwidth}
      \begin{block}{Exemples d'indicateurs}
        \begin{itemize}
          \item Effectifs par promotion (BUT1 / BUT2 / BUT3).
          \item Taux de réussite et distributions de notes.
          \item Suivi de progression (par semestre / UE / module).
        \end{itemize}
      \end{block}
      \begin{block}{Sources}
        \begin{itemize}
          \item API ScoDoc (notes, bulletins, jurys, absences).
        \end{itemize}
      \end{block}
    \column{0.46\textwidth}
      \includegraphics[width=\textwidth]{scolarite.png}
  \end{columns}
\end{frame}

\subsection{Réussite \& alertes}

\begin{frame}{KPIs de cohorte (exemples de clés)}
  \small
  \begin{tabularx}{\textwidth}{@{}lX@{}}
    \toprule
    \textbf{Clé} & \textbf{Interprétation} \\
    \midrule
    \texttt{effectif\_total} & Nombre total d'étudiants de la cohorte \\
    \texttt{moyenne\_promo} & Moyenne générale de la promotion \\
    \texttt{ecart\_type} / \texttt{mediane} & Dispersion / valeur centrale \\
    \texttt{taux\_reussite} & \% avec moyenne $\ge 10$ \\
    \texttt{taux\_difficulte} & \% avec moyenne $< 8$ \\
    \texttt{taux\_excellence} & \% avec moyenne $\ge 16$ \\
    \bottomrule
  \end{tabularx}
  \vspace{0.6em}
  \footnotesize
  Détails complets : \texttt{docs/INDICATEURS.md}.
\end{frame}

\begin{frame}{Alertes individuelles (types \& seuils par défaut)}
  \footnotesize
  \begin{tabularx}{\textwidth}{@{}lXl@{}}
    \toprule
    \textbf{Type} & \textbf{Description} & \textbf{Seuil} \\
    \midrule
    \texttt{difficulte\_academique} & Moyenne générale basse & $< 8/20$ \\
    \texttt{assiduite} & Taux d'absences non justifiées élevé & $> 15\%$ \\
    \texttt{decrochage} & Score de risque de décrochage & $> 0{,}7$ \\
    \texttt{progression\_negative} & Chute vs semestre précédent & $> -2$ pts \\
    \texttt{retard\_travaux} & Travaux non rendus & $> 3$ \\
    \texttt{absence\_evaluation} & Absences aux évaluations & $> 2$ \\
    \bottomrule
  \end{tabularx}
  \vspace{0.6em}
  \small
  \begin{itemize}
    \item Les seuils sont configurables par département (API \texttt{/api/\{dept\}/alertes/config}).
    \item La fiche étudiant consolide notes, absences, progression et risques.
  \end{itemize}
\end{frame}

\subsection{Recrutement}
\begin{frame}{Module Recrutement (Parcoursup / eCandidat)}
  \begin{columns}[T]
    \column{0.54\textwidth}
      \begin{block}{Objectifs}
        \begin{itemize}
          \item Suivre une campagne (v\oe ux, confirmations, admissions).
          \item Analyser l'origine des candidats (type de bac, lycée, département).
          \item Produire des stats synthétiques et exportables.
        \end{itemize}
      \end{block}
      \begin{block}{Import}
        \begin{itemize}
          \item CSV (UTF-8, séparateur \texttt{;}) via module Upload.
        \end{itemize}
      \end{block}
    \column{0.46\textwidth}
      \includegraphics[width=\textwidth]{recrutement.png}
  \end{columns}
\end{frame}

\subsection{Budget}
\begin{frame}{Module Budget}
  \begin{columns}[T]
    \column{0.54\textwidth}
      \begin{block}{Ce que l'on suit}
        \begin{itemize}
          \item Budget annuel, lignes budgétaires, dépenses.
          \item Répartition par catégories (fonctionnement / investissement, etc.).
          \item Alertes sur seuils et consommation.
        \end{itemize}
      \end{block}
      \begin{block}{Administration}
        \begin{itemize}
          \item CRUD + imports dédiés (pages Admin Budget).
        \end{itemize}
      \end{block}
    \column{0.46\textwidth}
      \includegraphics[width=\textwidth]{budget.png}
  \end{columns}
\end{frame}

\subsection{EDT \& maquettes}
\begin{frame}{Module EDT \& Maquettes}
  \begin{columns}[T]
    \column{0.54\textwidth}
      \begin{block}{Indicateurs}
        \begin{itemize}
          \item Charge horaire par enseignant.
          \item Taux d'occupation des salles.
          \item Répartition CM/TD/TP et cohérence avec maquette.
        \end{itemize}
      \end{block}
    \column{0.46\textwidth}
      \includegraphics[width=\textwidth]{edt.png}
  \end{columns}
\end{frame}

\section{Architecture \& clés techniques}

\begin{frame}{Architecture (vue simplifiée)}
  \begin{block}{Chaîne de traitement}
    \centering
    \textbf{React/Vite} \ $\rightarrow$ \ \textbf{API FastAPI} \ $\rightarrow$ \ \textbf{Adapters} \ $\rightarrow$ \ \textbf{Sources} \\
    \vspace{0.3em}
    \textbf{API FastAPI} \ $\leftrightarrow$ \ \textbf{PostgreSQL/SQLite} \quad et \quad \textbf{Redis (cache)}
  \end{block}
  \begin{itemize}
    \item Routes scindées par département : \texttt{/api/\{department\}/...}
    \item Adapters : encapsulent ScoDoc, Parcoursup, Excel/CSV (pandas).
  \end{itemize}
\end{frame}

\begin{frame}{Backend : points clés}
  \begin{itemize}
    \item FastAPI + Pydantic v2, routes par domaine (\texttt{app/api/routes/}).
    \item Modèles SQLAlchemy + migrations Alembic.
    \item Cache Redis avec TTL par domaine et option \texttt{?refresh=true}.
    \item Jobs planifiés (APScheduler) pour pré-charger/rafraîchir des données.
  \end{itemize}
  \vspace{0.5em}
  \small
  Références : \texttt{AGENTS.md}, \texttt{DOCUMENTATION.md}.
\end{frame}

\begin{frame}{Frontend : points clés}
  \begin{itemize}
    \item React 18 + TypeScript + Vite.
    \item Tailwind CSS + Recharts pour la dataviz.
    \item Data fetching via TanStack Query (cache côté client, refetch, staleTime).
    \item Pages métier : \texttt{frontend/src/pages/} ; composants réutilisables : \texttt{frontend/src/components/}.
  \end{itemize}
\end{frame}

\begin{frame}{Sécurité : authentification \& permissions}
  \begin{columns}[T]
    \column{0.56\textwidth}
      \begin{itemize}
        \item CAS (production) ou mock (dev) \(\rightarrow\) JWT pour les requêtes API.
        \item Permissions par département : voir / modifier par domaine + droits d'import/export.
        \item Rôle \textit{admin département} : tous droits sur le département.
      \end{itemize}
      \vspace{0.4em}
      \footnotesize
      Écrans : \texttt{admin\_users.png}, \texttt{admin\_permissions.png}.
    \column{0.44\textwidth}
      \includegraphics[height=0.34\textheight]{admin_users.png}
      \vspace{0.4em}
      \includegraphics[height=0.34\textheight]{admin_permissions.png}
  \end{columns}
\end{frame}

\begin{frame}[fragile]{Clés de configuration (.env)}
  \small
  \begin{block}{Variables principales}
\begin{verbatim}
DEBUG=true|false
SECRET_KEY=... (>= 32 chars en prod)
DATABASE_URL=... (sinon SQLite)
REDIS_URL=redis://...
CAS_USE_MOCK=true|false
CAS_SERVER_URL=...
CAS_SERVICE_URL=...
FRONTEND_URL=...
SCODOC_BASE_URL=...
SCODOC_USERNAME=...
SCODOC_PASSWORD=...
SCODOC_DEPARTMENT=...
\end{verbatim}
  \end{block}
  \footnotesize
  Modèle : \texttt{.env.prod.example}.
\end{frame}

\begin{frame}[fragile]{Clés API (exemples d'endpoints)}
  \small
  \begin{block}{Endpoints (extraits)}
\begin{verbatim}
GET /api/{dept}/scolarite/indicators
GET /api/{dept}/alertes/
GET /api/{dept}/alertes/etudiant/{id}
GET /api/{dept}/indicateurs/tableau-bord
GET /api/{dept}/recrutement/...
GET /api/{dept}/budget/...
\end{verbatim}
  \end{block}
  \footnotesize
  Détails et structures : \texttt{docs/INDICATEURS.md}, \texttt{docs/SCODOC\_API.md}.
\end{frame}

\section{Exploitation}

\begin{frame}[fragile]{Déploiement (aperçu)}
  \begin{itemize}
    \item Dev : API + Front séparés (Vite + Uvicorn), Redis local.
    \item Prod : stack Docker (Nginx + Front + API + DB + Redis).
  \end{itemize}
  \vspace{0.5em}
  \begin{block}{Commandes usuelles}
\begin{verbatim}
# Dev
docker compose up -d redis
cd backend && alembic upgrade head
cd backend && uvicorn app.main:app --reload --port 8000 \
  --app-dir backend
cd frontend && npm run dev
\end{verbatim}
  \end{block}
\end{frame}

\begin{frame}[fragile]{Tests (backend)}
  \begin{block}{pytest}
\begin{verbatim}
cd backend
pytest -v
\end{verbatim}
  \end{block}
  \small
  \begin{itemize}
    \item Les routes métiers attendent un JWT + permissions.
    \item En tests, utiliser le login dev/mocks ou override des dépendances.
  \end{itemize}
\end{frame}

\section{Roadmap}

\begin{frame}{Prochaines étapes}
  \begin{itemize}
    \item Valider la connexion ScoDoc sur données réelles (perf, erreurs réseau).
    \item Durcir la production : secrets, HTTPS, monitoring, logs.
    \item Renforcer le suivi des jobs et du cache (alerting, observabilité).
  \end{itemize}
\end{frame}

\begin{frame}{Ressources}
  \begin{itemize}
    \item \texttt{README.md} (démarrage rapide, variables d'environnement)
    \item \texttt{DOCUMENTATION.md} (walkthrough avec captures)
    \item \texttt{docs/INDICATEURS.md} (KPIs réussite/alertes, endpoints)
    \item \texttt{docs/SCODOC\_API.md} (référence API ScoDoc 9)
  \end{itemize}
\end{frame}

\begin{frame}
  \centering
  \Huge Questions ?
\end{frame}

\end{document}
